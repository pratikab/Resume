\documentclass[]{deedy-resume-openfont}
\usepackage{amsmath}
\usepackage{fontawesome}
\begin{document}

\lastupdated
\pagenumbering{gobble}
\namesection{Pratik}{Bhangale}{
   \faEnvelope { } \href{mailto:bhangalepratik8@gmail.com}{bhangalepratik8@gmail.com} | 
\faMobile { } \href{tel:918960742030}{+91 8960 74 2030} |
\faGithub{ \href{https://github.com/pratikab}{Github}} | \faLinkedinSquare{ \href{https://www.linkedin.com/in/pratikab/}{LinkedIn}}  
}


\begin{minipage}[t]{0.30\textwidth} 

\section{Education} 

\subsection{\href{https://iitk.ac.in/}{IIT {} Kanpur}}
\descript{B. Tech. in Computer Science\\ Minor in Microelectronics}
\location{Jun 14 - Jul '18  \\ CPI: 7.6/10}
\sectionsep
\subsection{Tanwani Junior College}
\descript{Twelfth Grade}
\location{Grad. May '14 | Maharashtra Board}
\location{ Percentage: 87.54\% }
\sectionsep

\subsection{Tanwani English School}
\descript{Tenth Grade}
\location{Grad. May '12 | Maharashtra Board }
\location{ Percentage: 98.00\% }
\sectionsep

\section{Skills}
\descript{Programming:}
 \textbullet{}C / C++ \textbullet{}Python \textbullet{}Verilog \\
 \textbullet{}Javascript \textbullet{}Linux Shell \\
%  \textbullet{}x86 and MIPS assembly\\
 \vspace{1mm}
 \descript{Libraries and Tools:}
  \textbullet{}MATLAB  \textbullet{}Tensorflow \textbullet{}Keras\\ \textbullet{}PyTorch \textbullet{}OpenCV \textbullet{} NLTK\\ \textbullet{}Django \textbullet{} GIT \textbullet{}Perforce \textbullet{}Makefile \\ \textbullet{}LATEX \textbullet{}GDB \textbullet{}Altium PCB designer \\
  \vspace{1mm}
  \descript{Operating Systems:}
 \textbullet{}Linux(Arch/Debian) \textbullet{}Windows
\sectionsep

\section{Coursework}
\textbullet{} Natural Language Processing (A)\\
\textbullet{} Introduction to  Machine learning (A)\\
\textbullet{} Computer Architecture (A)\\
\textbullet{} Computer Systems and Security (A)\\
\textbullet{} Computer Networks\\
\textbullet{} Compiler Techniques \\
\textbullet{} Database Management Systems \\
\textbullet{} Data Structures \& Algorithms \\
\textbullet{} Digital Electronics\\
\textbullet{} Introduction to Microelectronics\\
\sectionsep

\section{Positions Of Responsibilities}
\textbullet{} Coordinator, Electronics Club IITK\\
\hspace{1.5cm}\location{Mar '16 - Mar '17}
\textbullet{} Sub-head, Team Robocon IITK\\
\hspace{1.5cm}\location{Jul '15 - Mar '16}
\sectionsep


\section{Interests}
\textbullet{} Computer Systems\\
\textbullet{} Machine Learning and AI\\
\textbullet{} Microelectronics
\sectionsep



\end{minipage} 
\hfill
\begin{minipage}[t]{0.69\textwidth} 

\section{Professional Experience}
\runsubsection{Software Engineer}
\location{\href{https://research.samsung.com/sri-n}{\textbf{Samsung Research Institute, Noida}}  \hfill{} June '18-Present}
\vspace{3.5mm}
\begin{tightemize}
\item Part of \textbf{Android Bring Up} team, responsible for Android OS boot on mobile hardware and software upgrades.
\item Finding and Resolving boot anomalies in Android bootloader and kernel.
\end{tightemize}
\runsubsection{Software Intern}
\location{\href{https://research.samsung.com/sri-n}{\textbf{Samsung Research Institute, Noida}}\hfill{}  May '17 - July '17}
\begin{tightemize}
\item Developed two factor authentication system for Fingerprint called \textbf{BioHashing}.
\item Detected core-points in a Fingerprint Image using Gradient Orientation Map and hashed it to multi-dimensional vector to generate a BioHash.
% \item Tested the system on FVC2002 dataset and achieved 4.34\% False Rejection Rate (FRR) at 0.0017\% False Acceptance Rate (FAR) and proved that it can be deployed.
\end{tightemize} 
\section{achievements}
\vspace{2mm}
\begin{tightemize}
	\item Cleared JEE Mains and Advanced 2014 with percentile 99.98
	\item Recipient of Kishor Vaigyanik Protsahan Yojana (\href{http://www.kvpy.iisc.ernet.in/main/index.htm}{KVPY}) scholarship since 2014 by Indian Institute Science, Banglore
% 	\item Appeared in Indian National Chemistry Olympiad 2014 among top 300 students from all over the India
	\item Recipient of National Talent Search Examination (\href{http://www.ncert.nic.in/programmes/talent_exam/index_talent.html}{NTSE}) scholarship by NCERT 
	\item Second Runners Up in \href{http://www.roboconindia.com/}{ABU Robocon India} 2016 among 64 participating universities.
	\item First place in Electromania, \href{https://techkriti.org/}{Techkriti} (Annual Technical Festival of IIT Kanpur)
\end{tightemize}

\section{Projects}
\runsubsection{deCAPTCHA}
\location{\textbf{Prof. Purushottam Kar, IITK} | \faGithub{ \href{https://github.com/pratikab/deCAPTCHA}{source}} \hfill{} July '17 - Nov '17}

\begin{tightemize}
\item Objective was to build efficient algorithms to break squirrel mail captchas using deep nets.
\item Used K-means clustering and Image Processing tools to overcome heavy noises and clutters cutting through captcha text that makes it difficult for CNN models to work,
\item Implemented CNN models from scratch for character recognition which proved to be 98\% accurate for character recognition and 85\% for entire captcha.
\end{tightemize}

\runsubsection{Machine Comprehension}\\\hspace{0.5cm}
\location{\textbf{Prof. Harish Karnick, IITK} | \faGithub{ \href{https://github.com/2ashish/NLP-Answering-Reading-Comprehension}{source}} \hfill{} Dec '17 - Apr '18}

\begin{tightemize}
\item Objective was to help machine understand the comprehension and answer questions.
\item Character embeddings of comprehensions and questions were passed into Bi-Directional LSTM layers to extract information vector.
\item Question Attention Layer was generated to find out important information in question, which later is multiplied with context to get answer from. 
\item Model proved to be 97\% effective on bAbI dataset by Facebook .
\end{tightemize}

\runsubsection{Computer Architecture}\\\hspace{0.5cm}
\location{\textbf{Prof. Mainak Chaudhuri, IITK} | \faGithub{ \href{https://github.com/Naruto8/Project422}{source}} \hfill{} Dec '16 - Apr '17}

\begin{tightemize}
\item Designed and simulated advanced state-of-the-art cache replacement policies like Least Recently Used(LRU), Static Re-reference Interval Prediction(SRRIP) and Signature-based Hit Predictor (SHiP) using Pintool by Intel on 16 set-associative L3 Cache.
\item Implemeted Pipelined MIPS processor with Register Forwarding to avoid data hazards on KSIM simulator.
\end{tightemize}

\runsubsection{C to x86 compiler}
\location{\textbf{Prof. Amey Karkare, IITK} | \faGithub{ \href{https://github.com/pratikab/Compiler-C-to-X86}{source}} \hfill{} Dec '16 - Apr '17}

\begin{tightemize}
\item Developed a full fledged Compiler for a subset of C language for x86 architecture in python.
\item Implemented Lexical Analyzer, Parser, 3 address code generator and final assembly code generator.
\item The compiler supported basic arithmetic, conditionals, mutual recursion, parameterized functions, global decla-
rations and scope handling.
\end{tightemize}


% \runsubsection{ABU Robocon '16}\\
% \location{\textbf{\href{http://iitk.ac.in/new/bhaskar-dasgupta}{Prof. Bhaskar Dasgupta}} | IITK \hfill{} Sept '15 - Mar '16}
% \begin{tightemize}
% \item Designed \& fabricated one autonomous and one semi-autonomous robots which could complete tasks like pole climbing, object placing, line and wall following to participate in ABU Robocon 2016 competition
% \item Secured Second Runner Up place among 105 colleges participating all over the India
% \end{tightemize}

% \runsubsection{FPGA based Audio Mixer}\\
% \location{\textbf{\href{http://students.iitk.ac.in/eclub}{Electronics Club}} | IITK \hfill{} May '15 - June '16}
% \begin{tightemize}
% \item Developed an Audio Mixer device using FPGA taking Audio signal as input and producing different sound effects like Echo, Pitch change, etc.
% \item Worked on parallel processing, serial communication (SPI) on Spartan 6 MOJO FPGA, using Verilog as Hardware Description Language
% \end{tightemize}

\runsubsection{Mini Projects}
\begin{tightemize}
\item Hotel automation system using Django and SQlite.
\item File Sharing with user authentication system on Networks.
\item OS process sceduling, system calls, page allocation implementation for NachOS.
\item Chat Bot with slack and hubot integration to do personalized tasks like setting reminders, fetching emails, adding notes and setting up team meetings.
\end{tightemize}

% \section{Positions of responsibilities} 

% \runsubsection{Coordinator, Electronics Club IITK}
% \location{\hfill{} March '16 - Present}
% \begin{tightemize}
% \item Conducted lecture series and workshop to introduce campus community to several hardwares like Micro-controllers, FPGAs, Sensors, etc.
% \item Mentored six Summer Projects under Electronic Club in Summer '16
% \item Raised fund and Initiated the use Printed Circuit Boards \& new fabrication methods like Waterjet Cutting.  
% \end{tightemize}
% \runsubsection{Sub-head, Team Robocon IITK}
% \location{\hfill{} Sept '15 - March '16}
% \begin{tightemize}
% \item Co-ordinated a Team of 10 juniors to create autonomous \& semi-autonomous robots
% \item Organized Mini-projects and Introductory lectures on PCB designing, Micro-controllers and Electronic Sensors
% \end{tightemize}

\end{minipage} 
\end{document}  \documentclass[]{article}